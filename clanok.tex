\documentclass{article}
\usepackage{graphicx} % Required for inserting images

\title{Recommendation Systems}
\author{Martin Bilka}
\date{September 2024}

\begin{document}

\maketitle

\section{Úvod}

Vzdelávanie je nám prirodzený ľudský proces ktorý každý jednotlivý z nás prežíva. Jedná sa o
fyziologický proces ktorého konaním nadobúdame znalosť a skúsenosti v relevatnej oblastí záujmu.
Preto milióny ludi každodenne niakou formou vyhľadávaju rôzne informácie, prihlasujú sa na rôzne
vzdelávacie platformy a rovíjajú svoje schopnosti. Čo je vďaka moderným technológiam a širokej
dostupnosti a prístupu k sieti, umožnilo vzniknutie nových vzdelávacích platforiem ako sú napríklad
platforma Udemy. Táto vzdelávacia platforma poskytuje široké množstvo vzdelávacieho materialu,
praktických úloh a cvičení z rôzných oblastí vedy, technológie, inžinierstva a matematiky, vrátane
obsahu liberálnych vied, tento obsah je zverejnený v rôznej podobe a od roznych inštruktorov čo z
tejto platformy robí dokumentový informačný systém, podľa samotnej platformy Udemy potvredené
na základe štatistický informácii ktoré detailujú 73 miliónov študentov, 250 tisíc kurzov generovanýchvyše 75 tisícmi inštruktormi, vyučivaných vo viac ako 75 jazykoch, čo robí z tejto platformy informačný systém monumentálnych rozmerov, postavený na filtrovaní veľkým množstvom
parametrov na základe ktorých sa uživateľ može rozhodnuť prehliadať alebo dohladať špecifický
material, čo vyžadovalo vytvorenie odporúčacieho systému ktorý odporúčí každému správny material
alebo kurz alebo inštruktora na základe drobných detailov pre najoptimálnejšie odporúčania, vrátane
zakompovania spätnej väzby pre najoptimálnejšie odporúčania a návrhy.
 
\end{document}